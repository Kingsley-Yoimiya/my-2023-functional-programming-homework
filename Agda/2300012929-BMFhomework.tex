\documentclass[UTF-8]{ctexart}
\usepackage{graphicx}
\usepackage{subfigure}
\usepackage{xcolor}
\usepackage{amsmath}
\usepackage{amssymb}
\usepackage{tabularx}
\usepackage{amssymb}
\usepackage{amsthm}	

\begin{document}

Author: 尹锦润

Student ID: 2300012929





\section{BMF1-1}

(1)

If $\displaystyle ( \otimes c) \cdotp \oplus /=\oplus /\cdotp ( \otimes c) \ *\ $ holds, then according to the defination, \ $\displaystyle ( \otimes c) \cdotp \oplus /[ a,b] =\oplus /\cdotp ( \otimes c) \ *\ [ a,b]$ means $\displaystyle ( a\otimes b) \otimes c=( a\otimes c) \oplus ( b\otimes c)$.

If $\displaystyle ( a\oplus b) \otimes c=( a\otimes c) \oplus ( b\otimes c)$ holds, then we can prove $\displaystyle ( \otimes c) \cdotp \oplus /=\oplus /\cdotp ( \otimes c) \ *\ .$ step by step:
\begin{itemize}
	\item $\displaystyle ( \otimes c) \cdotp \oplus /[] =[] ,\oplus /\cdotp ( \otimes c) \ *\ []$
	\item $\displaystyle ( \otimes c) \cdotp \oplus /[ a] =a\otimes c,\oplus /\cdotp ( \otimes c) \ *\ [ a] =a\otimes c$
	\item If x satisfied the rule $\displaystyle ( \otimes c) \cdotp \oplus /x=\oplus /\cdotp ( \otimes c) \ *\ x$ ,then
	
	$\displaystyle \begin{aligned}
		& ( \otimes c) \cdotp \oplus /( x++[ a])\\
		= & ( \otimes c) *(( \oplus /x) \oplus a)\\
		& \{using\ ( a\oplus b) \otimes c=( a\otimes c) \oplus ( b\otimes c)\}\\
		= & (( \otimes c) \cdotp \oplus /x) \oplus \ ( a\otimes c))
	\end{aligned}$
	
	
	
	$\displaystyle \begin{aligned}
		& \oplus /\cdotp ( \otimes c) \ *\ ( x++[ a])\\
		= & ( \oplus /\cdotp ( \otimes c) \ *\ x) \oplus (( a\otimes c))\\
		= & (( \otimes c) \cdotp \oplus /x) \oplus (( a\otimes c))
	\end{aligned}$
\end{itemize}

so
\begin{center}
	$\displaystyle ( \otimes c) \cdotp \oplus /=\oplus /\cdotp ( \otimes c) \ *\ $ 
\end{center}

is equivalant to
\begin{center}
	$\displaystyle ( a\oplus b) \otimes c=( a\otimes c) \oplus ( b\otimes c)$
	
	
	
\end{center}
(2)

Because $\displaystyle f=\oplus /\cdotp \otimes /*\ tails=\odot \nrightarrow _{e}$ where $\displaystyle e=id_{\otimes } ,a\odot b=( a\otimes b) \oplus e$,then satisfiy the rules:
\begin{equation*}
	\begin{aligned}
		f[] & =e\\
		f( x++[ a]) & =\odot \nrightarrow _{\odot \nrightarrow _{e} \ x} a=f\ x\odot a
	\end{aligned}
\end{equation*}

\section{BMF1-2}

In the \texttt{mss.hs}.

\section{BMF1-3}


\begin{equation*}
	\begin{aligned}
		S= & \oplus /\cdotp f*\cdotp segs\\
		= & \oplus /\cdotp f*\cdotp ++/\cdotp tails\ *\cdotp \ inits\\
		= & \oplus /\cdotp ( \oplus /\cdotp f*\cdotp tails) *\cdotp \ inits\\
		= & \oplus /\cdotp ( h\cdotp \odot \nrightarrow _{e}) *\cdotp inits\\
		= & \oplus /\cdotp h*\odot \overline{\nrightarrow } \ e
	\end{aligned}
\end{equation*}
(I can't write ⊙→// e) 

\end{document}